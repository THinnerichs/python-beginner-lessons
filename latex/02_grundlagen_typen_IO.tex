% The Slide Definitions
\documentclass[10pt]{beamer}

%packages
\usepackage[ngerman]{babel}
\usepackage[utf8]{inputenc}
\usepackage{amsmath,tabu}
\usepackage{color}
\usepackage{tikz}
\usetikzlibrary{matrix,chains,positioning,decorations.pathreplacing,arrows}

\usetikzlibrary{fit,shapes}
\usetikzlibrary{calc,shadings}
\usepackage{pgfplots}
\usepackage{color}
\usepackage{colortbl}
\usepackage{eurosym}
\usepackage{mathtools}
\usepackage{listings}
\usepackage{tabularx}

%definitions
\usepackage{algorithm,algorithmic}

%theme
\usetheme{Dresden}
\usecolortheme{rose}
\useoutertheme{tree}

%environments
\newenvironment{ExampleGer}{\begin{exampleblock}{Example}}{\end{exampleblock}}

\newenvironment{customlegend}[1][]{%
	\begingroup
	% inits/clears the lists (which might be populated from previous
	% axes):
	\csname pgfplots@init@cleared@structures\endcsname
	\pgfplotsset{#1}%
}{%
	% draws the legend:
	\csname pgfplots@createlegend\endcsname
	\endgroup
}%

%definitions
\def\addlegendimage{\csname pgfplots@addlegendimage\endcsname}
% definition to insert numbers
\pgfkeys{/pgfplots/number in legend/.style={%
		/pgfplots/legend image code/.code={%
			\node at (0.295,-0.0225){#1};
		},%
	},
}


% color definitions
\definecolor{mygreen}{rgb}{0,0.6,0}
\definecolor{mygray}{rgb}{0.5,0.5,0.5}
\definecolor{mymauve}{rgb}{0.58,0,0.82}

\lstset{
    backgroundcolor=\color{white},
    % choose the background color;
    % you must add \usepackage{color} or \usepackage{xcolor}
    basicstyle=\footnotesize\ttfamily,
    % the size of the fonts that are used for the code
    breakatwhitespace=false,
    % sets if automatic breaks should only happen at whitespace
    breaklines=true,                 % sets automatic line breaking
    captionpos=b,                    % sets the caption-position to bottom
    commentstyle=\color{mygreen},    % comment style
    % deletekeywords={...},
    % if you want to delete keywords from the given language
    extendedchars=true,
    % lets you use non-ASCII characters;
    % for 8-bits encodings only, does not work with UTF-8
    frame=single,                    % adds a frame around the code
    keepspaces=true,
    % keeps spaces in text,
    % useful for keeping indentation of code
    % (possibly needs columns=flexible)
    keywordstyle=\color{blue},       % keyword style
    % morekeywords={*,...},
    % if you want to add more keywords to the set
    numbers=left,
    % where to put the line-numbers; possible values are (none, left, right)
    numbersep=5pt,
    % how far the line-numbers are from the code
    numberstyle=\tiny\color{mygray},
    % the style that is used for the line-numbers
    rulecolor=\color{black},
    % if not set, the frame-color may be changed on line-breaks
    % within not-black text (e.g. comments (green here))
    stepnumber=1,
    % the step between two line-numbers.
    % If it's 1, each line will be numbered
    stringstyle=\color{mymauve},     % string literal style
    tabsize=4,                       % sets default tabsize to 4 spaces
    % show the filename of files included with \lstinputlisting;
    % also try caption instead of title
    language = Python,
	showspaces = false,
	showtabs = false,
	showstringspaces = false,
	escapechar = ,
}

\def\ContinueLineNumber{\lstset{firstnumber=last}}
\def\StartLineAt#1{\lstset{firstnumber=#1}}
\let\numberLineAt\StartLineAt



\newcommand{\codeline}[1]{
	\alert{\texttt{#1}}
}


% Author and Course information
% This Document contains the information about this course.

% Authors of the slides
\author{Philipp Hanisch, Valentin Roland}

% Name of the Course
\institute{Python-Grundlagen}

% Fancy Logo 
\titlegraphic{\hfill\includegraphics[height=1.25cm]{fsr_logo_cropped}}



% Custom Bindings
% \newcommand{\codeline}[1]{
%	\alert{\texttt{#1}}
%}


% Presentation title
\title{Listen, Ein- und Ausgaben}
\date{18.11.2020}

\begin{document}
	
\maketitle

\begin{frame}{Letzte Woche: Zusammenfassung}
	\begin{itemize}
		\item rechnen in der Shell
		\item Indentation/Einrückung
		\item Kommentare
		\item Vergleichen von Werten
		\item Boolesche Werte (True, False, and, or)
		\item Kontrollstrukturen
		\begin{itemize}
			\item if, else, elif
			\item while und continue
		\end{itemize}
	\end{itemize}
\end{frame}


\begin{frame}{Gliederung}
    \setbeamertemplate{section in toc}[sections numbered]
    \tableofcontents
\end{frame}

\section{Intermezzo: Listen}

\begin{frame}{Was sind Listen?}
	Wie alltägliche Listen. In Python:
	\begin{itemize}
		\item \texttt{[]}, bzw. \texttt{[1,2,3]}
		\item Methoden: \texttt{insert(), append(), pop(), remove(), index()}
		\item siehe auch \texttt{help([])}
	\end{itemize}
\end{frame}

\begin{frame}[fragile]{Wie verwende ich Listen}
	\begin{lstlisting}
	a = [1,5,3,2] # create new list
	a.sort() # a is now sorted
	print (a[3]) # 4th element of sorted list -> 5
	a[3] = 7 # sets last/fourth(!) element to 7
	highest = a.pop() # removes last element (7) and returns it
	highest += 1
	a.insert(0, highest) # add 8 at the start
	print (a) # -> [8,1,2,3]
	
	\end{lstlisting}
\end{frame}

\section{For-Schleife}

\begin{frame}[fragile]{For-Schleife}
	Klassische Zählschleife gibt es nicht in Python\\
	$\hookrightarrow$ \texttt{for} durchläuft Listen
	\begin{lstlisting}
	a = [1,2,3]
	for elem in a:
		print (elem)
	\end{lstlisting}
\end{frame}

\begin{frame}[fragile]{Wie zählen?}
	\begin{lstlisting}
	for index in range(10):
		print (index)
		# prints numbers 0 to 9
	\end{lstlisting}
	\begin{lstlisting}
	a = [1,2,3]
	for index, elem in enumerate(a):
		print (index, elem)
	\end{lstlisting}
\end{frame}

\begin{frame}{Einige Aufgaben 3}
	Schreibe ein Python Programm, welches
	\begin{enumerate}
		\item alle Zahlen von 1 bis 20 addiert
		\item alle Zahlen einer Liste aufsummiert
		\item alle geraden Zahlen von 1 bis 20 ausgibt
		\item eine gegebene Liste in ihrer Reihenfolge invertiert
	\end{enumerate}
\end{frame}


% ---- Input von der Konsole
\section{Input}

\begin{frame}[fragile]{Problem}
	\begin{lstlisting}
	number = input('Gib eine Zahl ein: ')
	result = 5 + number
	print('Deine Zahl plus 5 ergibt: ' + result)
	\end{lstlisting}
\end{frame}

\begin{frame}[fragile]{Problem: Fehlerhafte Typen}
	\begin{lstlisting}
	number = input('Gib eine Zahl ein: ')
	result = 5 + number # Fehler: number ist ein String!
	print('Deine Zahl plus 5 ergibt: ' + result)
	\end{lstlisting}
\end{frame}



% ---- Typen und Typumwandlungen
\section{Typen}
\begin{frame}{Arbeiten mit Typen}
	\begin{itemize}
		\item dynamische Typisierung
		\item erlaubte Operationen
		\item Typumwandlung (Casting)
	\end{itemize}
\end{frame}

\begin{frame}{Rückblick: Typübersicht}
    \begin{tabular}{c|l}
		Name & Funktion \\ \hline
		\texttt{int} & Ganzzahl "beliebiger" Größe \\
		\texttt{float} & Kommazahl "beliebiger" Größe \\
		\texttt{str} & Zeichenkette \\
		\texttt{bool} & Wahrheitswert (\texttt{True}, \texttt{False})\\ \hline
		\texttt{list} & gewöhnliche Liste \\
		\texttt{tuple} & unveränderbares n-Tupel \\
		\texttt{set} & (mathematische) Menge von Objekten \\
		\texttt{dict} & Hash-Map \\
	\end{tabular}
\end{frame}

\begin{frame}[fragile]{Grundlagen: Typen}
	\begin{itemize}
		\item Typbestimmung: \texttt{type(...)}
		\item Casting: \texttt{int('42')}, \texttt{float('10.3')}, etc.
		\item potentielle Fehlerquelle
	\end{itemize}

	\begin{lstlisting}
	a = int('42')
	text = input('Deine Eingabe: ')

	b = int(text) # kann einen Fehler werfen

	3 + 6.3 # => 9.3
	int(3 + 6.3) # => 9

	def harm_folge(n):
		if type(n) == int:
			return 1/n
	\end{lstlisting}



\end{frame}

% ---- Konsolenausgabe & Stringformatierung
\section{Ausgabe}

\begin{frame}[fragile]{Konsolenausgabe}
	Ausgabe über \texttt{print(...)}

	\begin{lstlisting}
	print('Wundervoller Text')
	print('Noch', 'mehr', 'wundervoller', 'Text')

	day = input('Welcher Tag ist heute?')
	time = input('Wie spaet ist es?')
	print('Es ist ' + day + ', um ' + time + '.')
	\end{lstlisting}
\end{frame}

\begin{frame}[fragile]{Konsolenausgabe}
	Ausgabe über \texttt{print(...)}

	\begin{lstlisting}
	print('Wundervoller Text')
	print('Noch', 'mehr', 'wundervoller', 'Text')

	day = input('Welcher Tag ist heute?')
	time = input('Wie spaet ist es?')
	print('Es ist ' + day + ', um ' + time + '.')

	# oder schoener:
	print('Es ist {}, um {}.'.format(day, time))
	\end{lstlisting}
\end{frame}

\begin{frame}[fragile]{Stringformatierung}
	String können mit \texttt{'...'.format(...)} formatiert werden:

	\begin{lstlisting}
	number = 3
	item = 'Katzen'
	place = 'APB'

	# per Reihenfolge
	'Es gibt {} {} im {}!'.format(number, item, place)

	# per Name
	'Es gibt {amount} {things} im {where}!'.format(where=place, amount=10, things=item)

	# per Reihenfolge und Name
	'Es gibt {} {things} im {}!'.format(number, place, things='Koalas')
	\end{lstlisting}
\end{frame}


% ---- Aufgaben
\section{Aufgaben}
\begin{frame}{Aufgaben}
	\begin{enumerate}
	\item Fragt den Nutzer nach Vor- und Nachnamen und grüßt ihn!
	\item Lest zwei Zahlen ein und berechnet die Summe!
	\item Lest eine (vom Nutzer vorher festzulegende) Anzahl an Zahlen ein. Berechnet die Summe (Maximum, Mittelwert, ...) und gebt sie zusammen mit der Liste der Zahlen aus!
	\item Modifiziert euer Programm so, dass der Nutzer nach jeder Zahl gefragt wird, ob er weitere Zahlen eingeben möchte!
	\item Wählt eine beliebige natürliche Zahl. Lasst den Nutzer eure Zahl raten. Wenn er falsch geraten hat, gebt einen Hinweis $(>, <)$!
	\end{enumerate}
\end{frame}

% nothing to do from here on
\end{document}
