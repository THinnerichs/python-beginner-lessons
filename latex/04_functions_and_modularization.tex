% The Slide documentDefinitions
\documentclass[10pt]{beamer}

%packages
\usepackage[ngerman]{babel}
\usepackage[utf8]{inputenc}
\usepackage{amsmath,tabu}
\usepackage{color}
\usepackage{tikz}
\usetikzlibrary{matrix,chains,positioning,decorations.pathreplacing,arrows}

\usetikzlibrary{fit,shapes}
\usetikzlibrary{calc,shadings}
\usepackage{pgfplots}
\usepackage{color}
\usepackage{colortbl}
\usepackage{eurosym}
\usepackage{mathtools}
\usepackage{listings}
\usepackage{tabularx}

%definitions
\usepackage{algorithm,algorithmic}

%theme
\usetheme{Dresden}
\usecolortheme{rose}
\useoutertheme{tree}

%environments
\newenvironment{ExampleGer}{\begin{exampleblock}{Example}}{\end{exampleblock}}

\newenvironment{customlegend}[1][]{%
	\begingroup
	% inits/clears the lists (which might be populated from previous
	% axes):
	\csname pgfplots@init@cleared@structures\endcsname
	\pgfplotsset{#1}%
}{%
	% draws the legend:
	\csname pgfplots@createlegend\endcsname
	\endgroup
}%

%definitions
\def\addlegendimage{\csname pgfplots@addlegendimage\endcsname}
% definition to insert numbers
\pgfkeys{/pgfplots/number in legend/.style={%
		/pgfplots/legend image code/.code={%
			\node at (0.295,-0.0225){#1};
		},%
	},
}


% color definitions
\definecolor{mygreen}{rgb}{0,0.6,0}
\definecolor{mygray}{rgb}{0.5,0.5,0.5}
\definecolor{mymauve}{rgb}{0.58,0,0.82}

\lstset{
    backgroundcolor=\color{white},
    % choose the background color;
    % you must add \usepackage{color} or \usepackage{xcolor}
    basicstyle=\footnotesize\ttfamily,
    % the size of the fonts that are used for the code
    breakatwhitespace=false,
    % sets if automatic breaks should only happen at whitespace
    breaklines=true,                 % sets automatic line breaking
    captionpos=b,                    % sets the caption-position to bottom
    commentstyle=\color{mygreen},    % comment style
    % deletekeywords={...},
    % if you want to delete keywords from the given language
    extendedchars=true,
    % lets you use non-ASCII characters;
    % for 8-bits encodings only, does not work with UTF-8
    frame=single,                    % adds a frame around the code
    keepspaces=true,
    % keeps spaces in text,
    % useful for keeping indentation of code
    % (possibly needs columns=flexible)
    keywordstyle=\color{blue},       % keyword style
    % morekeywords={*,...},
    % if you want to add more keywords to the set
    numbers=left,
    % where to put the line-numbers; possible values are (none, left, right)
    numbersep=5pt,
    % how far the line-numbers are from the code
    numberstyle=\tiny\color{mygray},
    % the style that is used for the line-numbers
    rulecolor=\color{black},
    % if not set, the frame-color may be changed on line-breaks
    % within not-black text (e.g. comments (green here))
    stepnumber=1,
    % the step between two line-numbers.
    % If it's 1, each line will be numbered
    stringstyle=\color{mymauve},     % string literal style
    tabsize=4,                       % sets default tabsize to 4 spaces
    % show the filename of files included with \lstinputlisting;
    % also try caption instead of title
    language = Python,
	showspaces = false,
	showtabs = false,
	showstringspaces = false,
	escapechar = ,
}

\def\ContinueLineNumber{\lstset{firstnumber=last}}
\def\StartLineAt#1{\lstset{firstnumber=#1}}
\let\numberLineAt\StartLineAt



\newcommand{\codeline}[1]{
	\alert{\texttt{#1}}
}


% Author and Course information
% This Document contains the information about this course.

% Authors of the slides
\author{Philipp Hanisch, Valentin Roland}

% Name of the Course
\institute{Python-Grundlagen}

% Fancy Logo 
\titlegraphic{\hfill\includegraphics[height=1.25cm]{fsr_logo_cropped}}



% Custom Bindings
% \newcommand{\codeline}[1]{
%	\alert{\texttt{#1}}
%}


% Presentation title
\title{Funktionen und Modularisierung}
\date{\today}

\begin{document}

\maketitle

\begin{frame}{Rückblick}
    \begin{itemize}
        \item Datentypen: \texttt{int, str, list, set, dict, ...}
        \item Komperatoren: \texttt{==, !=, ...}
        \item Schleifen: \texttt{while, for}
        \item Ein-/Ausgabe: \texttt{input, print}
    \end{itemize}
\end{frame}

\section{Heute: Funktionen}

\begin{frame}{Funktionen}
    \begin{itemize}
        \item Eine \textbf{Funktion} ist ein in sich geschlossener Codeblock, der eine bestimmte Aktion ausführt.
        \item \textbf{Funktionen} können mit \textbf{Parametern} aufgerufen werden und können \textbf{Rückgabewerte} liefern.
        \item \textbf{Funktionen} erlauben es Code zu \alert{modularisieren}, zu \alert{organisieren} und \alert{wiederverwendbar} zu machen.
    \end{itemize}
\end{frame}

\begin{frame}[fragile]{Built-In Funktionen}
\begin{lstlisting}
# Die Anzahl der Elemente eines Containers bestimmen
numbers = [1, 2, 3, 4, 5]
len(numbers)

# Den Typ einer Variable herausfinden
n_float = 42.8
type(n_float)

# Informationen über ein Objekt bekommen
set_of_even_numbers = set(n for n in range(1, 10) if n % 2 == 0)
help(set_of_even_numbers)

# Weitere: sorted(), enumerate(), ...
\end{lstlisting}
\textbf{\href{https://docs.python.org/3/library/functions.html}{Hier findet ihr eine Liste der Python 3.9.1 Built-in Funktionen.}}
\end{frame}

\begin{frame}[fragile]{Eigene Funktionen I}
\begin{lstlisting}
# Funktionsdefiniton startet mit dem Keyword 'def'

def simple_function():

    # Hier beginnt der Körper (body) der Funktion

    print("The mitochondria is the powerhouse of the cell.")

    # Hier endet der Körper der Funktion
\end{lstlisting}
\begin{itemize}
    \item Kein \textbf{Parameter}, kein \textbf{Rückgabewert}
\end{itemize}
\end{frame}

\begin{frame}[fragile]{Eigene Funktionen II}
\begin{lstlisting}
# Der Wert eines übergebenen Parameters wird an eine
# Variable im Körper der aufgerufenen Funktion gebunden

def format_greeting(name):
    return f"Hello {name}. Nice to meet you again!"

# Der Rückgabewert einer Funktion kann
# in einer Variable gespeichert werden

result = format_greeting("Sun")
print(result)
\end{lstlisting}
\begin{itemize}
    \item Ein \textbf{Parameter}, ein \textbf{Rückgabewert}
\end{itemize}
\end{frame}

\begin{frame}[fragile]{Eigene Funktionen III}
\begin{lstlisting}
def calculate_price(item, price, amount=5):
    costs = amount * price
    return f"{amount} {items}'s will cost you {costs} $."
\end{lstlisting}
\begin{itemize}
    \item Funktionen lassen sich auf verschiedene Arten aufrufen.
\end{itemize}
\begin{lstlisting}
>>> calculate_price("Apple", 0.3, 10)
# 10 Apple's will cost you 3.0 $

>>> calculate_price(price=0.3, amount=10, item="Apple")
# 10 Apple's will cost you 3.0 $

>>> calculate_price("Apple", 0.3)
# 10 Apple's will cost you 1.5 $
\end{lstlisting}
\end{frame}

\begin{frame}[fragile]{Parametrisieren will gekonnt sein!}
Bei den Parametern einer Funktion unterscheidet man zwischen verschiedenen Typen.
\begin{itemize}
    \item \textbf{Positionelle} Parameter (Erforderliche Parameter)
    \item \textbf{Keyword} Parameter
    \item \textbf{Optionale} Parameter (Parameter mit Default Wert)
\end{itemize}
\end{frame}

\begin{frame}[fragile]{Mini-Exkurs: Documentation}
    \begin{lstlisting}
    def complicated_funktion(gamma, offset, inertia):
        """This function computes the result
        of calculation foobar for some given
        parameters.

        A function can get really complicated!
        But we can describe what a function does in
        plain text in the function docstring.
        """
        ...
    \end{lstlisting}
    \begin{itemize}
        \item \textbf{\alert{Code wird häufiger gelesen als geschrieben.}}
        \item \textbf{Dokumentiert euren Code} und ihr müssst weniger lesen!
        \item Gebt euren Funktionen und ihren Parametern \textbf{deskriptive Namen}!
    \end{itemize}
\end{frame}

\begin{frame}{Einige Aufgaben\texttrademark 6}
    \begin{enumerate}
    	\item Schreibe eine Funktion, welche zwei Zahlen addiert.
        \item Schreibe eine Funktion, welche die Anzahl der Vokale einer Zeichenkette berechnet.
        \item Schreibe eine Funktion, welche die Länge eines Videos im Format \texttt{mm:ss} (bspw. \texttt{02:54}) entgegen nimmt und die Länge des Videos in Sekunden zurückgibt.
        \item Schreibe eine Funktion, welche die \texttt{Fakultät} einer Ganzahl berechnet. Die Fakultät einer Ganzzahl ist das Produkt der Ganzzahl mit allen kleineren positiven Ganzzahlen.
        \item Schreibe eine Funktion, welche die \texttt{n-te} Zahl der \href{https://de.wikipedia.org/wiki/Fibonacci-Folge}{\texttt{Fibonacci-Folge}} berechnet.
    \end{enumerate}
\end{frame}

\begin{frame}{Ressourcen zum Üben/Informieren}
    \begin{itemize}
        \item \textbf{\href{https://realpython.com/defining-your-own-python-function/}{Real Python Artikel: Defining your own python function}}
        \item \textbf{\href{https://adventofcode.com/}{Advent of Code}}
    \end{itemize}
\end{frame}

\end{document}
