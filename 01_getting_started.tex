% The Slide Definitions
\documentclass[10pt]{beamer}

%packages
\usepackage[ngerman]{babel}
\usepackage[utf8]{inputenc}
\usepackage{amsmath,tabu}
\usepackage{color}
\usepackage{tikz}
\usetikzlibrary{matrix,chains,positioning,decorations.pathreplacing,arrows}

\usetikzlibrary{fit,shapes}
\usetikzlibrary{calc,shadings}
\usepackage{pgfplots}
\usepackage{color}
\usepackage{colortbl}
\usepackage{eurosym}
\usepackage{mathtools}
\usepackage{listings}
\usepackage{tabularx}

%definitions
\usepackage{algorithm,algorithmic}

%theme
\usetheme{Dresden}
\usecolortheme{rose}
\useoutertheme{tree}

%environments
\newenvironment{ExampleGer}{\begin{exampleblock}{Example}}{\end{exampleblock}}

\newenvironment{customlegend}[1][]{%
	\begingroup
	% inits/clears the lists (which might be populated from previous
	% axes):
	\csname pgfplots@init@cleared@structures\endcsname
	\pgfplotsset{#1}%
}{%
	% draws the legend:
	\csname pgfplots@createlegend\endcsname
	\endgroup
}%

%definitions
\def\addlegendimage{\csname pgfplots@addlegendimage\endcsname}
% definition to insert numbers
\pgfkeys{/pgfplots/number in legend/.style={%
		/pgfplots/legend image code/.code={%
			\node at (0.295,-0.0225){#1};
		},%
	},
}


% color definitions
\definecolor{mygreen}{rgb}{0,0.6,0}
\definecolor{mygray}{rgb}{0.5,0.5,0.5}
\definecolor{mymauve}{rgb}{0.58,0,0.82}

\lstset{
    backgroundcolor=\color{white},
    % choose the background color;
    % you must add \usepackage{color} or \usepackage{xcolor}
    basicstyle=\footnotesize\ttfamily,
    % the size of the fonts that are used for the code
    breakatwhitespace=false,
    % sets if automatic breaks should only happen at whitespace
    breaklines=true,                 % sets automatic line breaking
    captionpos=b,                    % sets the caption-position to bottom
    commentstyle=\color{mygreen},    % comment style
    % deletekeywords={...},
    % if you want to delete keywords from the given language
    extendedchars=true,
    % lets you use non-ASCII characters;
    % for 8-bits encodings only, does not work with UTF-8
    frame=single,                    % adds a frame around the code
    keepspaces=true,
    % keeps spaces in text,
    % useful for keeping indentation of code
    % (possibly needs columns=flexible)
    keywordstyle=\color{blue},       % keyword style
    % morekeywords={*,...},
    % if you want to add more keywords to the set
    numbers=left,
    % where to put the line-numbers; possible values are (none, left, right)
    numbersep=5pt,
    % how far the line-numbers are from the code
    numberstyle=\tiny\color{mygray},
    % the style that is used for the line-numbers
    rulecolor=\color{black},
    % if not set, the frame-color may be changed on line-breaks
    % within not-black text (e.g. comments (green here))
    stepnumber=1,
    % the step between two line-numbers.
    % If it's 1, each line will be numbered
    stringstyle=\color{mymauve},     % string literal style
    tabsize=4,                       % sets default tabsize to 4 spaces
    % show the filename of files included with \lstinputlisting;
    % also try caption instead of title
    language = Python,
	showspaces = false,
	showtabs = false,
	showstringspaces = false,
	escapechar = ,
}

\def\ContinueLineNumber{\lstset{firstnumber=last}}
\def\StartLineAt#1{\lstset{firstnumber=#1}}
\let\numberLineAt\StartLineAt



\newcommand{\codeline}[1]{
	\alert{\texttt{#1}}
}


% Author and Course information
% This Document contains the information about this course.

% Authors of the slides
\author{Philipp Hanisch, Valentin Roland}

% Name of the Course
\institute{Python-Grundlagen}

% Fancy Logo 
\titlegraphic{\hfill\includegraphics[height=1.25cm]{fsr_logo_cropped}}



% Custom Bindings
% \newcommand{\codeline}[1]{
%	\alert{\texttt{#1}}
%}


% Presentation title
\title{Kontrollstrukturen}
\date{\today}


\begin{document}

\maketitle

\begin{frame}{Gliederung}
    \setbeamertemplate{section in toc}[sections numbered]
    \tableofcontents
\end{frame}

% ---- Einführung Kontrollstrukturen ----
\section{Was sind Kontrollstrukturen?}
\begin{frame}{Wozu?}
    \begin{itemize}
        \item Strukturieren Programmverhalten
        \item für alle imperativen Sprachen ähnlich
        \item z.B.: Verzweigung, Zählschleife
    \end{itemize}
\end{frame}

\begin{frame}{Kontrollstrukturen in Python}
    Python kennt:
    \begin{itemize}
        \item Ein- / Zweiseitige Verzweigung: \texttt{if .. : / if .. : .. else: ..}
        \item Zählschleife: \texttt{for .. in .. :}
        \item Kopfgesteuerte Schleife: \texttt{while .. :}
    \end{itemize}
\end{frame}
% ----------------------- Der Python Interpreter ------------------------------
\section{Verzweigungen ("{}If-Statements"{})}
\begin{frame}[fragile]{Einseitige Verzweigungen}
    \begin{lstlisting}
    if a == b:
        print ("a is equal to b!")
    print ("this is printed every time. bye.")
    \end{lstlisting}
\end{frame}

\begin{frame}[fragile]{Zweiseitige Verzweigungen}
    \begin{lstlisting}
    if a == b:
        print ("a is equal to b!")
    else:
        print ("a is not equal to b!")
        # be careful: 1 == 1, but 1 != "1" !
    
    print ("this is printed every time. bye.")
    \end{lstlisting}
\end{frame}

\begin{frame}[fragile]{Verschachtelte Verzweigungen}
    \begin{lstlisting}
    if a < b:
        print ("a is less than b!")
    else:
        if a == b:
            print ("a == b!")
        print ("a >= b!")
        # is also executed when a == b! 
    \end{lstlisting}
\end{frame}

\begin{frame}[fragile]{Kurzform elif Verzweigungen}
    Oder mit \texttt{elif}:
    \begin{lstlisting}
    if a < b:
        print ("a is less than b!")
    elif a == b:
        print ("a == b!")
    else:
        print ("a > b!")
        # is not if when a == b! 
    \end{lstlisting}
\end{frame}

\section{While-Schleife}

\begin{frame}[fragile]{While-Schleife}
    $\rightarrow$ Wiederholt, solange Bedingung erfüllt ist:
    \begin{lstlisting}
    a = 10
    while a > -1:
        a -= 1
    \end{lstlisting}
\end{frame}

\begin{frame}[fragile]{Vorzeitiges Abbrechen} 
    \begin{lstlisting}
    a = 10
    while a > -1:
        a -= 1
        if a > 3:
            continue # jump to next interation
        print ("countdown:", a) # only printed for 3,2,1
        if a == 1:
            break # break out of loop immediately
    \end{lstlisting}
\end{frame}

\section{Intermezzo: Listen}

\begin{frame}{Was sind Listen?}
    Wie alltägliche Listen. In Python:
    \begin{itemize}
        \item \texttt{[]}, bzw. \texttt{[1,2,3]}
        \item Methoden: \texttt{insert(), append(), pop(), remove(), index()}
        \item siehe auch \texttt{help([])}
    \end{itemize}
\end{frame}

\begin{frame}[fragile]{Wie verwende ich Listen}
    \begin{lstlisting}
    a = [1,5,3,2] # create new list
    a.sort() # a is now sorted
    print (a[3]) # 4th element of sorted list -> 5
    highest = a.pop() # removes last element (5) and returns it
    highest += 1
    a.insert(0, highest) # add 6 at the start
    print (a) # -> [6,1,2,3]
    \end{lstlisting}
\end{frame}

\section{For-Schleife}

\begin{frame}[fragile]{For-Schleife}
    Klassische Zählschleife gibt es nicht in Python\\
    $\hookrightarrow$ \texttt{for} durchläuft Listen
    \begin{lstlisting}
    a = [1,2,3]
    for elem in a:
        print (elem)
    \end{lstlisting}
\end{frame}

\begin{frame}[fragile]{Wie zählen?}
    \begin{lstlisting}
    for index in range(10):
        print (index)
    # prints numbers 0 to 9
    \end{lstlisting}
    \begin{lstlisting}
    a = [1,2,3]
    for index, elem in enumerate(a):
        print (index, elem)
    \end{lstlisting}
\end{frame}

% nothing to do from here on
\end{document}
