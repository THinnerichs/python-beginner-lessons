% The Slide Definitions
\documentclass[10pt]{beamer}

%packages
\usepackage[ngerman]{babel}
\usepackage[utf8]{inputenc}
\usepackage{amsmath,tabu}
\usepackage{color}
\usepackage{tikz}
\usetikzlibrary{matrix,chains,positioning,decorations.pathreplacing,arrows}

\usetikzlibrary{fit,shapes}
\usetikzlibrary{calc,shadings}
\usepackage{pgfplots}
\usepackage{color}
\usepackage{colortbl}
\usepackage{eurosym}
\usepackage{mathtools}
\usepackage{listings}
\usepackage{tabularx}

%definitions
\usepackage{algorithm,algorithmic}

%theme
\usetheme{Dresden}
\usecolortheme{rose}
\useoutertheme{tree}

%environments
\newenvironment{ExampleGer}{\begin{exampleblock}{Example}}{\end{exampleblock}}

\newenvironment{customlegend}[1][]{%
	\begingroup
	% inits/clears the lists (which might be populated from previous
	% axes):
	\csname pgfplots@init@cleared@structures\endcsname
	\pgfplotsset{#1}%
}{%
	% draws the legend:
	\csname pgfplots@createlegend\endcsname
	\endgroup
}%

%definitions
\def\addlegendimage{\csname pgfplots@addlegendimage\endcsname}
% definition to insert numbers
\pgfkeys{/pgfplots/number in legend/.style={%
		/pgfplots/legend image code/.code={%
			\node at (0.295,-0.0225){#1};
		},%
	},
}


% color definitions
\definecolor{mygreen}{rgb}{0,0.6,0}
\definecolor{mygray}{rgb}{0.5,0.5,0.5}
\definecolor{mymauve}{rgb}{0.58,0,0.82}

\lstset{
    backgroundcolor=\color{white},
    % choose the background color;
    % you must add \usepackage{color} or \usepackage{xcolor}
    basicstyle=\footnotesize\ttfamily,
    % the size of the fonts that are used for the code
    breakatwhitespace=false,
    % sets if automatic breaks should only happen at whitespace
    breaklines=true,                 % sets automatic line breaking
    captionpos=b,                    % sets the caption-position to bottom
    commentstyle=\color{mygreen},    % comment style
    % deletekeywords={...},
    % if you want to delete keywords from the given language
    extendedchars=true,
    % lets you use non-ASCII characters;
    % for 8-bits encodings only, does not work with UTF-8
    frame=single,                    % adds a frame around the code
    keepspaces=true,
    % keeps spaces in text,
    % useful for keeping indentation of code
    % (possibly needs columns=flexible)
    keywordstyle=\color{blue},       % keyword style
    % morekeywords={*,...},
    % if you want to add more keywords to the set
    numbers=left,
    % where to put the line-numbers; possible values are (none, left, right)
    numbersep=5pt,
    % how far the line-numbers are from the code
    numberstyle=\tiny\color{mygray},
    % the style that is used for the line-numbers
    rulecolor=\color{black},
    % if not set, the frame-color may be changed on line-breaks
    % within not-black text (e.g. comments (green here))
    stepnumber=1,
    % the step between two line-numbers.
    % If it's 1, each line will be numbered
    stringstyle=\color{mymauve},     % string literal style
    tabsize=4,                       % sets default tabsize to 4 spaces
    % show the filename of files included with \lstinputlisting;
    % also try caption instead of title
    language = Python,
	showspaces = false,
	showtabs = false,
	showstringspaces = false,
	escapechar = ,
}

\def\ContinueLineNumber{\lstset{firstnumber=last}}
\def\StartLineAt#1{\lstset{firstnumber=#1}}
\let\numberLineAt\StartLineAt



\newcommand{\codeline}[1]{
	\alert{\texttt{#1}}
}


% Author and Course information
% This Document contains the information about this course.

% Authors of the slides
\author{Philipp Hanisch, Valentin Roland}

% Name of the Course
\institute{Python-Grundlagen}

% Fancy Logo 
\titlegraphic{\hfill\includegraphics[height=1.25cm]{fsr_logo_cropped}}



% Custom Bindings
% \newcommand{\codeline}[1]{
%	\alert{\texttt{#1}}
%}


% Presentation title
\title{Dictionaries}

\date{\today}


\begin{document}

\maketitle

\section{Rückblick: Listen}

\begin{frame}[fragile]{Listen}
    Liste von Werten: Zuordnung \emph{Index} $\rightarrow$ \emph{Wert} 
    \begin{lstlisting}
l = ["Hund", "Katze", "Maus"]

print (l[0]) # -> Hund
print (l[2]) # -> Maus
    \end{lstlisting}

    Zuordnung ist implizit:
    \begin{lstlisting}
for index, wert in enumerate(l):
    print (index, wert)

# -> 0, "Hund"
# -> 1, "Katze"
# -> 2, "Maus"
    \end{lstlisting}
\end{frame}

\section{Dictionaries}

\begin{frame}[fragile]{Dictionaries}
    Dictionary $\equiv$ \glqq Wörterbuch\grqq:\\
    \begin{itemize}
        \item Allgemeine Zuordnung \emph{Objekt} $\rightarrow$ \emph{Objekt}
        \item z.B. \emph{Spielerobjekt} $\rightarrow$ \emph{Punktestand} statt Liste 
    \end{itemize}
    \begin{lstlisting}
# Schlüssel: Wert
d = {
    "Hello": "World",
    "Hinz": "Kunz",
    "Pi": 3.1415926,
    42: True,
}
    \end{lstlisting}
\end{frame}

\begin{frame}[fragile]{Verwendung}
    \begin{lstlisting}
# Schlüssel: Wert
d = {
    "Hello": "World",
    "Hinz": "Kunz",
    "Pi": 3.1415926,
    42: True,
}

print(d["Hello"]) # -> World
print(d[42]) # -> True

print(d.get("Hello")) # -> World
print(d["Spam"]) # -> KeyError!
print(d.get("Spam")) # -> None

d["e"] = 2.71
print(d) 
# {'Hello': 'World', 'Hinz': 'Kunz',
#  'Pi': 3.1415926, 42: True, 'e': 2.71}
    \end{lstlisting}
\end{frame}

\begin{frame}[fragile]{Verwendung (II)}

    \begin{lstlisting}
# Schlüssel: Wert
spieler = {
    "Spieler 1": 1,
    "Spieler 2": 3,
    "Spieler 3": 2,
}

# Liste der Spieler ausgeben (ACHTUNG: beliebige Reihenfolge!)
print (spieler.keys()) 
# -> ["Spieler 1", "Spieler 2", "Spieler 3"]

# Liste der Punktstände (ACHTUNG: beliebige Reihenfolge!)
print (spieler.values()) # -> [1, 3, 2]

# Schleife über Einträge (ACHTUNG: beliebige Reihenfolge!)
for name, punkte in spieler.items():
    print ("{} hat momentan {} Punkte!".format(name, punkte))
# -> Spieler 1 hat momentan 1 Punkte!
# -> Spieler 2 hat momentan 3 Punkte!
# -> Spieler 3 hat momentan 2 Punkte!
    \end{lstlisting}
\end{frame}

\section{Vereinfachung der Spiel-Klasse: Live!}

\section{Aufgaben}

\begin{frame}{Aufgaben}
    \begin{itemize}
        \item Erstelle ein Dictionary mit der Zuordnung \emph{deutsches Wort} $\rightarrow$ \emph{englisches Wort} (10 Einträge). Schreibe ein Programm, dass für eine Liste von Wörtern alle in diesem Dictionary vorhandenen deutschen Wörter durch die englischen Wörter ersetzt.
        \item Erstelle ein Diagramm, in dem der Punktestand aller Spieler im zeitlichen Verlauf dargestellt ist (s. Beispiel in dieser Präsentation)
        \item Füge einen schummelnden Spieler zu deinem Spiel hinzu:
            \begin{enumerate}
                \item Mehr Würfel
                \item Andere Seitenzahl
                \item seid kreativ $\dots$
            \end{enumerate}
            Wie wirkt sich das Schummeln bei den verschiedenen Wertungsmodi aus? \\
            (\texttt{"wuerfelsumme"}, \texttt{"wuerfelmaximum"}, \texttt{"paschzahl"})\\
            Dies wird bei höherer Rundenzahl möglicherweise deutlicher.

    \end{itemize}
\end{frame}
\end{document}
