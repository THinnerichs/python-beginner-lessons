% The Slide Definitions
\documentclass[10pt]{beamer}

%packages
\usepackage[ngerman]{babel}
\usepackage[utf8]{inputenc}
\usepackage{amsmath,tabu}
\usepackage{color}
\usepackage{tikz}
\usetikzlibrary{matrix,chains,positioning,decorations.pathreplacing,arrows}

\usetikzlibrary{fit,shapes}
\usetikzlibrary{calc,shadings}
\usepackage{pgfplots}
\usepackage{color}
\usepackage{colortbl}
\usepackage{eurosym}
\usepackage{mathtools}
\usepackage{listings}
\usepackage{tabularx}

%definitions
\usepackage{algorithm,algorithmic}

%theme
\usetheme{Dresden}
\usecolortheme{rose}
\useoutertheme{tree}

%environments
\newenvironment{ExampleGer}{\begin{exampleblock}{Example}}{\end{exampleblock}}

\newenvironment{customlegend}[1][]{%
	\begingroup
	% inits/clears the lists (which might be populated from previous
	% axes):
	\csname pgfplots@init@cleared@structures\endcsname
	\pgfplotsset{#1}%
}{%
	% draws the legend:
	\csname pgfplots@createlegend\endcsname
	\endgroup
}%

%definitions
\def\addlegendimage{\csname pgfplots@addlegendimage\endcsname}
% definition to insert numbers
\pgfkeys{/pgfplots/number in legend/.style={%
		/pgfplots/legend image code/.code={%
			\node at (0.295,-0.0225){#1};
		},%
	},
}


% color definitions
\definecolor{mygreen}{rgb}{0,0.6,0}
\definecolor{mygray}{rgb}{0.5,0.5,0.5}
\definecolor{mymauve}{rgb}{0.58,0,0.82}

\lstset{
    backgroundcolor=\color{white},
    % choose the background color;
    % you must add \usepackage{color} or \usepackage{xcolor}
    basicstyle=\footnotesize\ttfamily,
    % the size of the fonts that are used for the code
    breakatwhitespace=false,
    % sets if automatic breaks should only happen at whitespace
    breaklines=true,                 % sets automatic line breaking
    captionpos=b,                    % sets the caption-position to bottom
    commentstyle=\color{mygreen},    % comment style
    % deletekeywords={...},
    % if you want to delete keywords from the given language
    extendedchars=true,
    % lets you use non-ASCII characters;
    % for 8-bits encodings only, does not work with UTF-8
    frame=single,                    % adds a frame around the code
    keepspaces=true,
    % keeps spaces in text,
    % useful for keeping indentation of code
    % (possibly needs columns=flexible)
    keywordstyle=\color{blue},       % keyword style
    % morekeywords={*,...},
    % if you want to add more keywords to the set
    numbers=left,
    % where to put the line-numbers; possible values are (none, left, right)
    numbersep=5pt,
    % how far the line-numbers are from the code
    numberstyle=\tiny\color{mygray},
    % the style that is used for the line-numbers
    rulecolor=\color{black},
    % if not set, the frame-color may be changed on line-breaks
    % within not-black text (e.g. comments (green here))
    stepnumber=1,
    % the step between two line-numbers.
    % If it's 1, each line will be numbered
    stringstyle=\color{mymauve},     % string literal style
    tabsize=4,                       % sets default tabsize to 4 spaces
    % show the filename of files included with \lstinputlisting;
    % also try caption instead of title
    language = Python,
	showspaces = false,
	showtabs = false,
	showstringspaces = false,
	escapechar = ,
}

\def\ContinueLineNumber{\lstset{firstnumber=last}}
\def\StartLineAt#1{\lstset{firstnumber=#1}}
\let\numberLineAt\StartLineAt



\newcommand{\codeline}[1]{
	\alert{\texttt{#1}}
}


% Author and Course information
% This Document contains the information about this course.

% Authors of the slides
\author{Philipp Hanisch, Valentin Roland}

% Name of the Course
\institute{Python-Grundlagen}

% Fancy Logo 
\titlegraphic{\hfill\includegraphics[height=1.25cm]{fsr_logo_cropped}}



% Custom Bindings
% \newcommand{\codeline}[1]{
%	\alert{\texttt{#1}}
%}


% Presentation title
\title{Arbeiten mit Dateien}
\date{\today}

\usepackage{tikz} \usetikzlibrary{arrows, positioning}

\tikzset{
  auto,
  invisible/.style={opacity=0},
  visible on/.style={alt={#1{}{invisible}}},
  alt/.code args={<#1>#2#3}{%
    \alt<#1>{\pgfkeysalso{#2}}{\pgfkeysalso{#3}} % \pgfkeysalso doesn't change the path
  },
}




\begin{document}

\maketitle

\begin{frame}{Gliederung}
    \setbeamertemplate{section in toc}[sections numbered]
    \tableofcontents
\end{frame}

\begin{frame}{Was zuletzt geschah}
	\begin{itemize}
		\item Module und wo sie zu finden sind
		\item Pakete (Packages): \texttt{import, from, *}
		\item boilerplate
	\end{itemize}
\end{frame}

% ---- Rückblick ----
\section{Motivation}

\begin{frame}{Wofür Dateien?}
	Python oft in Scripten für Datenverarbeitung:
    \begin{itemize}
        \item Messwerte
        \item Logs
        \item Konfigurationen
        \item ...
    \end{itemize}
    $\hookrightarrow$ liegen in Dateien, sollen in weitere Dateien
\end{frame}

\begin{frame}{Warum Python?}
	\begin{itemize}
        \item Einfache Dateiformate (z.B. csv) in Python sehr einfach nutzbar
        \item Betriebssystemunabhängig (keine Shell-Magie auf Windows)
        \item Mit Python3: Keine Unicodeprobleme...
	\end{itemize}
\end{frame}

\section{Lesen und Schreiben von Dateien}

\begin{frame}{Konzept}
	Grundlegender Ablauf:
    \begin{itemize}
        \item Öffnen $\rightarrow$ Datei-Objekt
        \item Lesen / Schreiben $\rightarrow$ Methoden auf Datei-Objekt
        \item Schließen $\rightarrow$ Ressourcen freigeben
    \end{itemize}
\end{frame}

\begin{frame}[fragile]{Dateien lesen}
    \begin{lstlisting}
# Datei öffnen
f = open("test.txt", "r")
# Sämtlichen Inhalt lesen
print (f.read())
# Datei schließen
f.close()\end{lstlisting}
    Auch nützlich zum Lesen\footnote{\url{https://www.tutorialspoint.com/python3/python_files_io.htm}}:
    \begin{lstlisting}
f = open("test.txt", "r")
# Liest bis zu 100 Zeichen (auch Zeilenumbrüche)
preview = f.read(100)
# Liest alle nachfolgenden! Zeilen in Liste
rest = f.readlines()
# z.B. ["Zeile 1", "Zeile 2", "Zeile 3"]
f.close()\end{lstlisting}
    $\hookrightarrow$ jedes lesen schiebt Cursor weiter!
\end{frame}

\begin{frame}[fragile]{Dateien schreiben}
    \begin{lstlisting}
# Datei öffnen
f = open("test.txt", "w")
f.write("Hello World")
f.close()
    \end{lstlisting}
    \begin{lstlisting}
f = open("test.txt", "w")
# Eine Zeile mit "Hello World" und Zeilenumbruch schreiben
f.write("Hello World\n")
# Danach weitere Zeilen anfügen
zeilen = "\n".join(["Zeile 1", "Zeile 2", "Zeile 3"])
f.write(zeilen)
f.close()
    \end{lstlisting}
\end{frame}




\begin{frame}[fragile]{Dateien richtig lesen}
\begin{lstlisting}
f = open("test.txt", "r")
# Liest bis zu 100 Zeichen (auch Zeilenumbrüche)
preview = f.read(100)
# Liest alle nachfolgenden! Zeilen in Liste
rest = f.readlines()
# z.B. ["Zeile 1", "Zeile 2", "Zeile 3"]
f.close()
\end{lstlisting}
\pause
\begin{lstlisting}
with open('test.txt', 'r') as f:
	for line in f:
		# trennt nach jedem '\n'
		# "Hallo Mama, Ich bin in einer Textdatei!\n"
		greetings, message = line.strip().split(',')
		...
\end{lstlisting}
\end{frame}

\begin{frame}[fragile]{Dateien richtig schreiben}
\begin{lstlisting}
f = open("test.txt", "w")
# Eine Zeile mit "Hello World" und Zeilenumbruch schreiben
f.write("Hello World\n")
# Danach weitere Zeilen anfügen
zeilen = "\n".join(["Zeile 1", "Zeile 2", "Zeile 3"])
f.write(zeilen)
f.close()
\end{lstlisting}
\pause
\begin{lstlisting}
with open(file='test.txt', 'w') as f:
	# entweder 
	f.write("Hallo Kind, toll gemacht!\n")
	# oder 
	print("Hallo Kind, toll gemacht!\n", file=f)
\end{lstlisting}
\end{frame}

\begin{frame}[fragile]{Was man sonst noch mit Dateien machen kann}
\begin{itemize}
	\item[r] - Read - Default value. Opens a file for reading, error if the file does not exist
	
	\item[a] - Append - Opens a file for appending, creates the file if it does not exist
	
	\item[w] - Write - Opens a file for writing, creates the file if it does not exist
	
	\item[x] - Create - Creates the specified file, returns an error if the file exists
	
\end{itemize}
Zusätzlich:
\begin{itemize}
	\item[t] - Text - Default value. Text mode
	
	\item[b] - Binary - Binary mode for e.g. images, pickled (See pickle package) files
	
\end{itemize}

\end{frame}

\begin{frame}[fragile]{Ein Beispiel}
\begin{lstlisting}
with open("demofile.txt", "rt") as f:
	# der standardfall, muss nicht explizit genannt werden
	...
\end{lstlisting}

\begin{lstlisting}
...
import pickle

spieler = Spieler(wuerfel, name)
spieler_list.append(spieler)

with open(file='spielerlist.pkl', mode='wb') as f:
	pickle.dump(spieler_liste, f)
\end{lstlisting}
\end{frame}

% --------------------- Aufgaben -------------------------

\section{Aufgaben}

\begin{frame}[fragile]{Einige Aufgaben}
    \begin{enumerate}
  		\item Downloaded \href{https://loremipsum.de/downloads/version5.txt}{https://loremipsum.de/downloads/version5.txt} von \url{loremipsum.de}
  		\item Zähle die Wörter in diesem File
  		\item Kopiere den Inhalt in eine andere Datei
  		\item Lösche alle \textbf{new line character} in diesem neuen file.
  	\end{enumerate}    
\end{frame}

\begin{frame}{Encodingszähmen leicht gemacht}
	\begin{itemize}
		\item Character und Zahlen werden intern durch Bitsequenzen repräsentiert
		\pause
		\item Gegeben eine Bitsequenz, wie soll diese interpretiert werden?
		\pause
		\item Header, Dateiendungen, Anderes Wissen
		\item \texttt{UTF-8, ASCII, UTF-16, ...}
		\item \texttt{Hex, Binary, Decimal}
	\end{itemize}
\end{frame}

\begin{frame}[fragile]
\begin{lstlisting}
>>> int('11', base=16)
17
>>> int('11', base=2)
3
>>> hex(17)
'0x11'
>>> bin(11)
'0b11'
\end{lstlisting}

\begin{lstlisting}
with open(file='test', mode='r', encoding='utf8'):
	...
\end{lstlisting}
\end{frame}

\begin{frame}[fragile]{\texttt{ord} und \texttt{chr}}
\texttt{chr(i)}\\
Return the string representing a character whose Unicode code point is the integer i. For example, chr(97) returns the string 'a', while chr(8364) returns the string '€'. This is the inverse of ord().
\begin{lstlisting}
>>> ord('a')
97
>>> chr(97)
'a'
>>> ord('A')
65
\end{lstlisting}
\end{frame}

\begin{frame}{Einige Aufgaben}
\begin{enumerate}
	\item Schreibe ein Programm, welches eine Caesar Chiffre beschreibt. (\texttt{caesar\_chiffre(text, offset)})
	\item Schreibe eine Liste aller geraden Quadratzahlen bis 1000 in eine Datei und lese diese Liste von der Datei wieder ein
	\item Lese die Dokumentation von \texttt{pickle}, und schreibe ein Python Programm, welches ein Spielerobject abspeichert und lädt. 
\end{enumerate}
\end{frame}

\end{document}
