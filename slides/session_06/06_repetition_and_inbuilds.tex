% The Slide Definitions
\documentclass[10pt]{beamer}

%packages
\usepackage[ngerman]{babel}
\usepackage[utf8]{inputenc}
\usepackage{amsmath,tabu}
\usepackage{color}
\usepackage{tikz}
\usetikzlibrary{matrix,chains,positioning,decorations.pathreplacing,arrows}

\usetikzlibrary{fit,shapes}
\usetikzlibrary{calc,shadings}
\usepackage{pgfplots}
\usepackage{color}
\usepackage{colortbl}
\usepackage{eurosym}
\usepackage{mathtools}
\usepackage{listings}
\usepackage{tabularx}

%definitions
\usepackage{algorithm,algorithmic}

%theme
\usetheme{Dresden}
\usecolortheme{rose}
\useoutertheme{tree}

%environments
\newenvironment{ExampleGer}{\begin{exampleblock}{Example}}{\end{exampleblock}}

\newenvironment{customlegend}[1][]{%
	\begingroup
	% inits/clears the lists (which might be populated from previous
	% axes):
	\csname pgfplots@init@cleared@structures\endcsname
	\pgfplotsset{#1}%
}{%
	% draws the legend:
	\csname pgfplots@createlegend\endcsname
	\endgroup
}%

%definitions
\def\addlegendimage{\csname pgfplots@addlegendimage\endcsname}
% definition to insert numbers
\pgfkeys{/pgfplots/number in legend/.style={%
		/pgfplots/legend image code/.code={%
			\node at (0.295,-0.0225){#1};
		},%
	},
}


% color definitions
\definecolor{mygreen}{rgb}{0,0.6,0}
\definecolor{mygray}{rgb}{0.5,0.5,0.5}
\definecolor{mymauve}{rgb}{0.58,0,0.82}

\lstset{
    backgroundcolor=\color{white},
    % choose the background color;
    % you must add \usepackage{color} or \usepackage{xcolor}
    basicstyle=\footnotesize\ttfamily,
    % the size of the fonts that are used for the code
    breakatwhitespace=false,
    % sets if automatic breaks should only happen at whitespace
    breaklines=true,                 % sets automatic line breaking
    captionpos=b,                    % sets the caption-position to bottom
    commentstyle=\color{mygreen},    % comment style
    % deletekeywords={...},
    % if you want to delete keywords from the given language
    extendedchars=true,
    % lets you use non-ASCII characters;
    % for 8-bits encodings only, does not work with UTF-8
    frame=single,                    % adds a frame around the code
    keepspaces=true,
    % keeps spaces in text,
    % useful for keeping indentation of code
    % (possibly needs columns=flexible)
    keywordstyle=\color{blue},       % keyword style
    % morekeywords={*,...},
    % if you want to add more keywords to the set
    numbers=left,
    % where to put the line-numbers; possible values are (none, left, right)
    numbersep=5pt,
    % how far the line-numbers are from the code
    numberstyle=\tiny\color{mygray},
    % the style that is used for the line-numbers
    rulecolor=\color{black},
    % if not set, the frame-color may be changed on line-breaks
    % within not-black text (e.g. comments (green here))
    stepnumber=1,
    % the step between two line-numbers.
    % If it's 1, each line will be numbered
    stringstyle=\color{mymauve},     % string literal style
    tabsize=4,                       % sets default tabsize to 4 spaces
    % show the filename of files included with \lstinputlisting;
    % also try caption instead of title
    language = Python,
	showspaces = false,
	showtabs = false,
	showstringspaces = false,
	escapechar = ,
}

\def\ContinueLineNumber{\lstset{firstnumber=last}}
\def\StartLineAt#1{\lstset{firstnumber=#1}}
\let\numberLineAt\StartLineAt



\newcommand{\codeline}[1]{
	\alert{\texttt{#1}}
}


% Author and Course information
% This Document contains the information about this course.

% Authors of the slides
\author{Philipp Hanisch, Valentin Roland}

% Name of the Course
\institute{Python-Grundlagen}

% Fancy Logo 
\titlegraphic{\hfill\includegraphics[height=1.25cm]{fsr_logo_cropped}}



% Custom Bindings
% \newcommand{\codeline}[1]{
%	\alert{\texttt{#1}}
%}


% Presentation title
\title{Collections}
\date{26.11.2020 und 03.12.2020}


\begin{document}
	
\maketitle

\begin{frame}{Was zuletzt geschah...}
	\begin{itemize}
		\item Datentypen: \texttt{int, str, list, dict, \ldots}
		\item Komperatoren: \texttt{==, !=, \ldots}
		\item Kontrollstrukturen: \texttt{if, elif, else}
		\item Schleifen: \texttt{while, for}
		\item \textbf{\alert{Funktionen: \texttt{return, def f(x): \ldots}}}
	\end{itemize}
\end{frame}

\begin{frame}{Was zuletzt geschah... 2}
	\begin{itemize}
		\item Objektorientiertes programmieren: \texttt{class, \_\_init\_\_, get}
		\item modellieren mit python (Wie man richtig würfelt)
	\end{itemize}
\end{frame}

\begin{frame}{Gliederung}
	\setbeamertemplate{section in toc}[sections numbered]
	\tableofcontents
\end{frame}

\begin{frame}{Quiz zu Typen}
	\footnotesize
	\begin{tabular}{c|l|l|l|l}
		Name & Funktion &iterierbar &anwendbare Funktionen &  Methoden\\ 
		\hline
		\texttt{int} & &&& \\
		\texttt{float} &  &&& \\
		\texttt{str} &  &&& \\
		\texttt{bool} &  &&&\\ \hline
		\texttt{list} &  &&& \\
		\texttt{tuple} & &&& \\
		\texttt{set} &  &&& \\
		\texttt{dict} &  &&& \\
	\end{tabular}
\end{frame}

\begin{frame}{Auflösung}
	\footnotesize
	\begin{tabular}{c|p{2cm}|l|l|l}
		Name & Funktion &iterierbar &anwendbare Funktionen &  Methoden\\ 
		\hline
		\texttt{int} & Ganzzahl &no& +, float&\texttt{to\_bytes}\\
		\texttt{float} & Kommazahl&no& +&\texttt{\_\_round\_\_}\\
		\texttt{str} & Zeichenkette &no& +, \texttt{len}&\texttt{lower()}\\
		\texttt{bool} & Wahrheitswert &no&\texttt{int}&$\dots$\\ 
		\hline
		\texttt{list} & gewöhnliche Liste &yes&\texttt{len}&\texttt{append}\\
		\texttt{tuple} & unveränderbares n-Tupel&yes&\texttt{len}&\texttt{count} \\
		\texttt{set} & (mathematische) Menge &yes&\texttt{len}&\texttt{pop, add, update}\\
		\texttt{dict} & Hash-Map &yes&\texttt{len}&\texttt{update, pop}\\
	\end{tabular}
\end{frame}

\begin{frame}{Einige Aufgaben 7}
	Schreibe eine Python Klasse/ein Python Programm, welche
	\begin{enumerate}
		\item ein Rechteck darstellt, dabei Höhe und Breite einliest, get und set, und eine Methode zur Berechnung der Fläche hat
		\item eine/-n VolleyballspielerIn darstellt (name, Nummer)
		\item welches eine Volleyballmannschaft darstellt (Mannschaft, Geld)
		\item eine Funktion, welche Transfers zwischen zwei Mannschaften ausführt
		\item zwei Mannschaften gegeneinander spielen lässt und ein zufälliges Ergebnis ausführt (Optional: Das Ergebnis nach Geld gewichten)
		\item eine Saison von 6 Mannschaften mit Hin- und Rückrunde simuliert
	\end{enumerate}
\end{frame}

\begin{frame}{Built-In functions}
	Bekannte Built-In Funktionen
	\begin{itemize}
		\item \texttt{int, float, str}
		\item \texttt{len, range, $\dots$}
	\end{itemize}
	\pause
	Neue und spannende Built-In Funktionen:
	\begin{itemize}
		\item \texttt{enumerate}
		\item \texttt{map}
		\item \texttt{lambda}
		\item \texttt{zip}
		\item \texttt{filter}
	\end{itemize}
\end{frame}

\begin{frame}[fragile]{enumerate}
\texttt{enumerate} nimmt iterable und gibt tuple von Index und Element zurück
\begin{lstlisting}
for i in range(len(my_list)-1):
	ele = my_list[i]
	f(ele, i%dim1)

for i, ele in enumerate(L):
	next\_ele = my\_list[i+1]
	f(ele, i%dim1)
\end{lstlisting}
\end{frame}

\begin{frame}[fragile]{map und lambda}
\begin{lstlisting}
def f(a,b):
	return a%b + 42
	
f = lambda a,b: a%b + 42
\end{lstlisting}
\pause 
\begin{lstlisting}
# map(func, iterable) -> Funktion auf alle Elemente

#use the list() function to display a readable version of the result:
list(map(lambda a: a+5, list_of_numbers))

\end{lstlisting}
\end{frame}

\begin{frame}[fragile]{filter und zip}
\begin{lstlisting}
ages = [5, 12, 17, 18, 24, 32]

def myFunc(x):
	if x < 18:
		return False
	else:
		return True

adults = filter(myFunc, ages)

for x in adults:
	print(x) 
\end{lstlisting}
\pause
\begin{lstlisting}
>>> a = ("John", "Charles", "Mike")
>>> b = ("Jenny", "Christy", "Monica", "Vicky")

>>> x = zip(a, b)

#use the list() function to display a readable version of the result:

>>> list(x)
[('John', 'Jenny'), ('Charles', 'Christy'), ('Mike', 'Monica')]

\end{lstlisting}
\end{frame}

\begin{frame}
	\begin{itemize}
		\item \texttt{enumerate}: ele -> i, ele
		\item \texttt{map}: map(func, iter) -> func(i for i in iter)
		\item \texttt{zip}: zip(iter1, iter2) -> iter of tuples
		\item \texttt{filter}: filter(func, iter) -> [iter for i in iter if func(i)]
	\end{itemize}
\end{frame}

\begin{frame}{Einige Aufgaben 8}
	Schreibe ein Python Programm, welches
	\begin{enumerate}
		\item eine jedes Element einer gegebenen Liste verdoppelt
		\item alle geraden Elemente eine gegebenen Liste von Zahlen zurückgibt
		\item 1-dimensionale Listen von Koordinaten nimmt und diese zu 3D-Koordinaten zusammenfügt
		\item 3 gleichlange Listen von Zahlen Elementweise addiert
		\item die Unterschiede zweier gegebener Listen ausgibt
	\end{enumerate}
\end{frame}






\end{document}