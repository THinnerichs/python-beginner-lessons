% The Slide Definitions
\documentclass[10pt]{beamer}

%packages
\usepackage[ngerman]{babel}
\usepackage[utf8]{inputenc}
\usepackage{amsmath,tabu}
\usepackage{color}
\usepackage{tikz}
\usetikzlibrary{matrix,chains,positioning,decorations.pathreplacing,arrows}

\usetikzlibrary{fit,shapes}
\usetikzlibrary{calc,shadings}
\usepackage{pgfplots}
\usepackage{color}
\usepackage{colortbl}
\usepackage{eurosym}
\usepackage{mathtools}
\usepackage{listings}
\usepackage{tabularx}

%definitions
\usepackage{algorithm,algorithmic}

%theme
\usetheme{Dresden}
\usecolortheme{rose}
\useoutertheme{tree}

%environments
\newenvironment{ExampleGer}{\begin{exampleblock}{Example}}{\end{exampleblock}}

\newenvironment{customlegend}[1][]{%
	\begingroup
	% inits/clears the lists (which might be populated from previous
	% axes):
	\csname pgfplots@init@cleared@structures\endcsname
	\pgfplotsset{#1}%
}{%
	% draws the legend:
	\csname pgfplots@createlegend\endcsname
	\endgroup
}%

%definitions
\def\addlegendimage{\csname pgfplots@addlegendimage\endcsname}
% definition to insert numbers
\pgfkeys{/pgfplots/number in legend/.style={%
		/pgfplots/legend image code/.code={%
			\node at (0.295,-0.0225){#1};
		},%
	},
}


% color definitions
\definecolor{mygreen}{rgb}{0,0.6,0}
\definecolor{mygray}{rgb}{0.5,0.5,0.5}
\definecolor{mymauve}{rgb}{0.58,0,0.82}

\lstset{
    backgroundcolor=\color{white},
    % choose the background color;
    % you must add \usepackage{color} or \usepackage{xcolor}
    basicstyle=\footnotesize\ttfamily,
    % the size of the fonts that are used for the code
    breakatwhitespace=false,
    % sets if automatic breaks should only happen at whitespace
    breaklines=true,                 % sets automatic line breaking
    captionpos=b,                    % sets the caption-position to bottom
    commentstyle=\color{mygreen},    % comment style
    % deletekeywords={...},
    % if you want to delete keywords from the given language
    extendedchars=true,
    % lets you use non-ASCII characters;
    % for 8-bits encodings only, does not work with UTF-8
    frame=single,                    % adds a frame around the code
    keepspaces=true,
    % keeps spaces in text,
    % useful for keeping indentation of code
    % (possibly needs columns=flexible)
    keywordstyle=\color{blue},       % keyword style
    % morekeywords={*,...},
    % if you want to add more keywords to the set
    numbers=left,
    % where to put the line-numbers; possible values are (none, left, right)
    numbersep=5pt,
    % how far the line-numbers are from the code
    numberstyle=\tiny\color{mygray},
    % the style that is used for the line-numbers
    rulecolor=\color{black},
    % if not set, the frame-color may be changed on line-breaks
    % within not-black text (e.g. comments (green here))
    stepnumber=1,
    % the step between two line-numbers.
    % If it's 1, each line will be numbered
    stringstyle=\color{mymauve},     % string literal style
    tabsize=4,                       % sets default tabsize to 4 spaces
    % show the filename of files included with \lstinputlisting;
    % also try caption instead of title
    language = Python,
	showspaces = false,
	showtabs = false,
	showstringspaces = false,
	escapechar = ,
}

\def\ContinueLineNumber{\lstset{firstnumber=last}}
\def\StartLineAt#1{\lstset{firstnumber=#1}}
\let\numberLineAt\StartLineAt



\newcommand{\codeline}[1]{
	\alert{\texttt{#1}}
}


% Author and Course information
% This Document contains the information about this course.

% Authors of the slides
\author{Philipp Hanisch, Valentin Roland}

% Name of the Course
\institute{Python-Grundlagen}

% Fancy Logo 
\titlegraphic{\hfill\includegraphics[height=1.25cm]{fsr_logo_cropped}}



% Custom Bindings
% \newcommand{\codeline}[1]{
%	\alert{\texttt{#1}}
%}


% Presentation title
\title{Praxis: Dijkstra-Algorithmus}
\date{\today}

\usepackage{tikz} \usetikzlibrary{arrows, positioning}

\tikzset{
  auto,
  invisible/.style={opacity=0},
  visible on/.style={alt={#1{}{invisible}}},
  alt/.code args={<#1>#2#3}{%
    \alt<#1>{\pgfkeysalso{#2}}{\pgfkeysalso{#3}} % \pgfkeysalso doesn't change the path
  },
}




\begin{document}

\maketitle

\begin{frame}{Gliederung}
    \setbeamertemplate{section in toc}[sections numbered]
    \tableofcontents
\end{frame}

% ---- Rückblick ----
\section{Problemstellung}

\begin{frame}{Das Problem}
	\textbf{gegeben:} 
	\begin{itemize}
		\item ein gerichteter, kantenbewerteter Graph $G = (V, E, d)$
		\item ein Startknoten S
	\end{itemize}
	\textbf{gesucht:} \\
	\begin{itemize}
		\item kürzeste Wege zu den anderen Knoten
	\end{itemize}

\end{frame}


\begin{frame}{Das Problem}
	\begin{figure}[h]
		\begin{tikzpicture}[>=stealth',baseline=-0.75ex,black,auto,node distance=4.5cm]
			\node[circle, white, draw, fill=blue] (1) {1};
			\node[circle, white, draw, above right = 2cm and 2cm of 1, fill=red] (2) {2};
			\node[circle, white, draw, below right = 2cm and 2cm of 1, fill=red] (3) {3};
			\node[circle, white, draw, right of = 1, fill=red] (4) {4};
			\node[circle, white, draw, right of = 2, fill=red] (5) {5};
			\node[circle, white, draw, right of = 3, fill=red] (6) {6};
			\node[circle, white, draw, right of = 4, fill=red] (7) {7};

			\path[->] (1) edge node {4} (2);
			\path[->] (1) edge node {1} (3);
			\path[->] (1) edge node {4} (4);

			\path[->] (2) edge [bend left=20] node {3} (4);
			\path[->] (2) edge node {3} (5);

			\path[->] (3) edge node {2} (4);
			\path[->] (3) edge node {7} (6);

			\path[->] (4) edge [bend left=20] node {2} (2);
			\path[->] (4) edge [bend left=20] node {5} (5);
			\path[->] (4) edge node {1} (6);

			\path[->] (5) edge [bend left=20] node {3} (4);
			\path[->] (5) edge [bend left=20] node {2} (7);
			
			\path[->] (6) edge node {4} (5);
			\path[->] (6) edge node {3} (7);
			
			\path[->] (7) edge [bend left=20] node {1} (5);
		\end{tikzpicture}
	\end{figure}
\end{frame}



\begin{frame}{Modellierung}
	\begin{itemize}
		\item Knotenmenge als Liste \\
		hier: \texttt{[1, 2, 3, 4, 5, 6, 7]}
		\item Kantenbewertung als Dictonary: (Knoten, Knoten) \rightarrow{} Zahl
		hier: \texttt{\{(1,2):4, (1,3):1, (1,4):4, (2,4):3, (2,5):3, (3,4):2, ...\}}
	\end{itemize}

\end{frame}

\section{Der Dijkstra-Algorithmus}

\begin{frame}{Die Idee}
	\begin{enumerate}
		\item \textbf{Initialisierung:}
		\item \textbf{Schleife:} 
			\begin{itemize}
				\item Wähle erreichbaren Knoten mit minimaler Entfernung
				\item Markiere Knoten als besucht
				\item Aktualisiere die Information der anderen Knoten
			\end{itemize}
		\item \textbf{Ausgabe}
	\end{enumerate}

\end{frame}


\begin{frame}{Beispiel}
	\begin{figure}[h]
		\vspace{-0.2cm}
		\begin{tikzpicture}[>=stealth',baseline=-0.75ex,black,auto,node distance=4.5cm]
			\node[circle, white, draw, fill=blue] (1) {1};
			\node[circle, white, draw, above right = 1cm and 2cm of 1, fill=red] (2) {2};
			\node[circle, white, draw, below right = 1cm and 2cm of 1, fill=red] (3) {3};
			\node[circle, white, draw, right of = 1, fill=red] (4) {4};
			\node[circle, white, draw, right of = 2, fill=red] (5) {5};
			\node[circle, white, draw, right of = 3, fill=red] (6) {6};
			\node[circle, white, draw, right of = 4, fill=red] (7) {7};

			\path[->] (1) edge node {4} (2);
			\path[->] (1) edge node {1} (3);
			\path[->] (1) edge node {4} (4);

			\path[->] (2) edge [bend left=20] node {3} (4);
			\path[->] (2) edge node {3} (5);

			\path[->] (3) edge node {2} (4);
			\path[->] (3) edge node {7} (6);

			\path[->] (4) edge [bend left=20] node {2} (2);
			\path[->] (4) edge [bend left=20] node {5} (5);
			\path[->] (4) edge node {1} (6);

			\path[->] (5) edge [bend left=20] node {3} (4);
			\path[->] (5) edge [bend left=20] node {2} (7);
			
			\path[->] (6) edge node {4} (5);
			\path[->] (6) edge node {3} (7);
			
			\path[->] (7) edge [bend left=20] node {1} (5);
		\end{tikzpicture}
	\end{figure}

	\begin{tabular}{c c c c c c c | c c c c c c c}
		\only<2->{$l_1$ & $l_2$ & $l_3$ & $l_4$ & $l_5$ & $l_6$ & $l_7$ & $p_1$ & $p_2$ & $p_3$ & $p_4$ & $p_5$ & $p_6$ & $p_7$ \\ \hline}
		\only<3->{0 & \infty & \infty & \infty & \infty & \infty & \infty & - & - & - & - & - & - & - \\}
		\only<4->{* & 4 & 1 & 4 & \infty & \infty & \infty & - & 1 & 1 & 1 & - & - & - \\}
		\only<5->{* & 4 & * & 3 & \infty & 8 & \infty & - & 1 & 1 & 3 & - & 3 & - \\}
		\only<6->{* & 4 & * & * & 8 & 4 & \infty & - & 1 & 1 & 3 & 4 & 4 & - \\}
		\only<7->{* & * & * & * & 7 & 4 & \infty & - & 1 & 1 & 3 & 2 & 4 & - \\}
		\only<8->{* & * & * & * & 7 & * & 7 & - & 1 & 1 & 3 & 2 & 4 & 6 \\}
		\only<9->{* & * & * & * & * & * & 7 & - & 1 & 1 & 3 & 2 & 4 & 6 \\}
		\only<10>{* & * & * & * & * & * & * & - & 1 & 1 & 3 & 2 & 4 & 6 \\}
	\end{tabular}
\end{frame}


\begin{frame}{Bezeichner}
	\begin{itemize}
		\item Graph $G = (V, E, d)$
		\item Menge $K$ an erreichbaren Knoten
		\item Startknoten $S$
		\item bisher kürzeste Entfernungen $l(k)$ für Knoten $k \in V$
		\item derzeitiger Vorgänger $p(k)$ für Knoten $k \in V$
	\end{itemize}
\end{frame}


\begin{frame}{Der Dijkstra-Algorithmus}
	\begin{enumerate}
		\item \textbf{Initialisierung:}
			\begin{itemize}
				\item $l(k) = \infty$ für alle Knoten $k \in V \setminus \{S\}$
				\item $l(S) = 0$
				\item $p(k) = $ \texttt{None} für alle Knoten $k \in V$ 
				\item $K = \{S\}$
			\end{itemize}
		\item \textbf{Schleife:} Solange $K \neq \varnothing$ \dots
			\begin{itemize}
				\item Wähle einen erreichbaren Knoten $v$ mit minimaler Entfernung: \\
					$v \in K$ mit $l(v) = min\{l(k) : k \in K\}$
				\item Markiere Knoten $v$ als besucht: \\
					$K := K \setminus \{v\}$
				\item Aktualisiere die Information der anderen Knoten $k \in V$: \\
					Wenn $l(v) + d(v,k) < l(k)$: \\ 
					dann setze $l(k) := l(v) + d(v,k)$ und $p(k) := v$ \\
					Füge $k$ ggf. zu $K$ hinzu
			\end{itemize}
		\item \textbf{Ausgabe}
	\end{enumerate}

\end{frame}



% --------------------- Aufgaben -------------------------

\section{Aufgaben}

\begin{frame}{Aufgaben}
	Implementiert den Dijkstra-Algorithmus!
\end{frame}

\end{document}
