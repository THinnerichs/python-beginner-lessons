% The Slide Definitions
\documentclass[10pt]{beamer}

%packages
\usepackage[ngerman]{babel}
\usepackage[utf8]{inputenc}
\usepackage{amsmath,tabu}
\usepackage{color}
\usepackage{tikz}
\usetikzlibrary{matrix,chains,positioning,decorations.pathreplacing,arrows}

\usetikzlibrary{fit,shapes}
\usetikzlibrary{calc,shadings}
\usepackage{pgfplots}
\usepackage{color}
\usepackage{colortbl}
\usepackage{eurosym}
\usepackage{mathtools}
\usepackage{listings}
\usepackage{tabularx}

%definitions
\usepackage{algorithm,algorithmic}

%theme
\usetheme{Dresden}
\usecolortheme{rose}
\useoutertheme{tree}

%environments
\newenvironment{ExampleGer}{\begin{exampleblock}{Example}}{\end{exampleblock}}

\newenvironment{customlegend}[1][]{%
	\begingroup
	% inits/clears the lists (which might be populated from previous
	% axes):
	\csname pgfplots@init@cleared@structures\endcsname
	\pgfplotsset{#1}%
}{%
	% draws the legend:
	\csname pgfplots@createlegend\endcsname
	\endgroup
}%

%definitions
\def\addlegendimage{\csname pgfplots@addlegendimage\endcsname}
% definition to insert numbers
\pgfkeys{/pgfplots/number in legend/.style={%
		/pgfplots/legend image code/.code={%
			\node at (0.295,-0.0225){#1};
		},%
	},
}


% color definitions
\definecolor{mygreen}{rgb}{0,0.6,0}
\definecolor{mygray}{rgb}{0.5,0.5,0.5}
\definecolor{mymauve}{rgb}{0.58,0,0.82}

\lstset{
    backgroundcolor=\color{white},
    % choose the background color;
    % you must add \usepackage{color} or \usepackage{xcolor}
    basicstyle=\footnotesize\ttfamily,
    % the size of the fonts that are used for the code
    breakatwhitespace=false,
    % sets if automatic breaks should only happen at whitespace
    breaklines=true,                 % sets automatic line breaking
    captionpos=b,                    % sets the caption-position to bottom
    commentstyle=\color{mygreen},    % comment style
    % deletekeywords={...},
    % if you want to delete keywords from the given language
    extendedchars=true,
    % lets you use non-ASCII characters;
    % for 8-bits encodings only, does not work with UTF-8
    frame=single,                    % adds a frame around the code
    keepspaces=true,
    % keeps spaces in text,
    % useful for keeping indentation of code
    % (possibly needs columns=flexible)
    keywordstyle=\color{blue},       % keyword style
    % morekeywords={*,...},
    % if you want to add more keywords to the set
    numbers=left,
    % where to put the line-numbers; possible values are (none, left, right)
    numbersep=5pt,
    % how far the line-numbers are from the code
    numberstyle=\tiny\color{mygray},
    % the style that is used for the line-numbers
    rulecolor=\color{black},
    % if not set, the frame-color may be changed on line-breaks
    % within not-black text (e.g. comments (green here))
    stepnumber=1,
    % the step between two line-numbers.
    % If it's 1, each line will be numbered
    stringstyle=\color{mymauve},     % string literal style
    tabsize=4,                       % sets default tabsize to 4 spaces
    % show the filename of files included with \lstinputlisting;
    % also try caption instead of title
    language = Python,
	showspaces = false,
	showtabs = false,
	showstringspaces = false,
	escapechar = ,
}

\def\ContinueLineNumber{\lstset{firstnumber=last}}
\def\StartLineAt#1{\lstset{firstnumber=#1}}
\let\numberLineAt\StartLineAt



\newcommand{\codeline}[1]{
	\alert{\texttt{#1}}
}


% Author and Course information
% This Document contains the information about this course.

% Authors of the slides
\author{Philipp Hanisch, Valentin Roland}

% Name of the Course
\institute{Python-Grundlagen}

% Fancy Logo 
\titlegraphic{\hfill\includegraphics[height=1.25cm]{fsr_logo_cropped}}



% Custom Bindings
% \newcommand{\codeline}[1]{
%	\alert{\texttt{#1}}
%}


% Presentation title
\title{Collections}
\date{26.11.2020 und 03.12.2020}


\begin{document}
	
\maketitle

\begin{frame}{Was zuletzt geschah...}
	\begin{itemize}
		\item Listen und wo sie zu finden sind (\texttt{help(...), [], L.append(...)})
		\item for-Schleifen (\texttt{for i in range(10):})
		\item Standarddatentypen und Umwandlung dieser (\texttt{int("42")})
		\item Eingabe und Ausgabe (\texttt{input("Zahlen, bitte:")})
		\item Strings (\texttt{\"abc" + "def", 'abc'[:-1]},)
	\end{itemize}
\end{frame}

\begin{frame}{Gliederung}
	\setbeamertemplate{section in toc}[sections numbered]
	\tableofcontents
\end{frame}

% ---- Was ist Objektorientierung ----
\section{Übung macht den Meister}

\begin{frame}{Einige Aufgaben\texttrademark 4.0}
	Schreibe ein Programm, welches
	\begin{enumerate}
		\item alle geraden Zahlen einer Zahlenliste in einer zweiten Liste speichert und stoppt, sobald eine $237$ vorkommt
		\item alle Elemente einer Liste von Strings zusammenfügt und ausgibt
		\item als Input den Namen und Heimatplaneten des Nutzers abfragt und diesen nett grüßt
		\item die Seitenlänge $h$ und zugehörige Höhe $h_c$ eines Dreiecks einliest und den Flächeninhalt ausgibt
		\item die Quersumme einer Zahl berechnet
	\end{enumerate}
\end{frame}

\section{Dictionaries}
\begin{frame}[fragile]{Nachtrag zu Listen: List comprehensions}
\begin{lstlisting}
    L = []
    for num in list_of_numbers:
        if num%2 == 0:
            L.append(num+1)
\end{lstlisting}
\begin{lstlisting}
    L = [num+1 for num in list_of_numbers if num%2==0]
\end{lstlisting}
\end{frame}

\begin{frame}[fragile]{Tupel und Mengen}
	\begin{itemize}
		\item[tuple] Listen mit festgesetzter Länge
		\begin{itemize}
			\item kein \texttt{append}, etc.
			\item normaler Zugriff und Veränderung einzelner Elemente wie bei lists
		\end{itemize}
		\item[set] mathematische Menge
		\begin{itemize}
			\item keine Duplikate
			\item ungeordnet
		\end{itemize}
	\end{itemize}
\begin{lstlisting}
>>> a=(1,2)
>>> a[1]
2
>>> s = {1,2,3,1}
>>> s
{1,2,3}
>>> s[0]
Error
>>> {1,2,3} & {2,3,4}
{2,3}
>>> {1,2,3} | {2,3,4}
{1,2,3,4}
>>> set(range(20))
...
\end{lstlisting}
\end{frame}

\begin{frame}{Einige Aufgaben\texttrademark 5.0}
	Schreibe ein Programm, welches
	\begin{enumerate}
		\item die Liste der Quadrate der Zahlen von 1 bis 20 erzeugt, wenn diese Zahlen kleiner als 5 oder größer als 14 sind (mit List comprehensions)
		\item eine gegebene, \textbf{geordnete} Liste von Zahlen von Duplikaten befreit
		\item[\textbf{Bonus:}] eine gegebene, \textbf{ungeordnete} Liste von Zahlen von Duplikaten befreit
	\end{enumerate}
\end{frame}

\begin{frame}[fragile]{Dictionaries}
	Dictionaries sind Mengen von key-value-pairs mit folgenden Eigenschaften
	\begin{itemize}
		\item unordered
		\item changeable
		\item ohne Duplikate (bezogen auf keys)
	\end{itemize}
\end{frame}

\begin{frame}[fragile]{Dictionaries}
\begin{lstlisting}
thisdict = {
    "brand": "Ford",
    "model": "Mustang",
    "year": 1964
}
print(thisdict)
print(thisdict['brand'])

thisdict['brand']='Trabant'
thisdict['color] # ERROR
thisdict.get('color')
\end{lstlisting}
\end{frame}

\begin{frame}[fragile]
Mengen erlauben keine Duplikate:
\begin{lstlisting}
thisdict = {
    "brand": "Ford",
    "model": "Mustang",
    "year": 1964,
    "year": 2020
}
print(thisdict) 

\end{lstlisting}	
\end{frame}

\begin{frame}[fragile]{Wie man über dicts iteriert}
dicts haben Methoden wie
\begin{itemize}
	\item \texttt{d.items()}
	\item \texttt{d.keys()}
	\item \texttt{d.values()}
\end{itemize}
\begin{lstlisting}
for key, value in d.items():
	d[key] = value+1
	
# dict comprehensions
d = {key: value+1 for (key,value) in d.items()}
\end{lstlisting}
\end{frame}

\begin{frame}{Rückblick: Typübersicht}
	\begin{tabular}{c|l}
		Name & Funktion \\ \hline
		\texttt{int} & Ganzzahl "beliebiger" Größe \\
		\texttt{float} & Kommazahl "beliebiger" Größe \\
		\texttt{str} & Zeichenkette \\
		\texttt{bool} & Wahrheitswert (\texttt{True}, \texttt{False})\\ \hline
		\texttt{list} & gewöhnliche Liste \\
		\texttt{tuple} & unveränderbares n-Tupel \\
		\texttt{set} & (mathematische) Menge von Objekten \\
		\texttt{dict} & Hash-Map \\
	\end{tabular}
\end{frame}

\begin{frame}{Einige Aufgaben\texttrademark 6}
Schreibe ein Python Programm, welches
	\begin{enumerate}
		\item zwei gegebene dicts zusammenfügt
		\item welches die Überschneidungen zweier gegebener dicts ausgibt
		\item ein gegebenes dict nach den values sortiert
		\item alle geraden Einträge eines gegebenen dicts löscht
		\item den größten und kleinsten value einer gegebenen Liste ausgibt
		\item den größten und kleinsten value eines gegebenen dicts ausgibt
		\item ein gegebenes dict nach values größer 170 filtert (Nur Paare mit $value>170$ verbleiben)
	\end{enumerate}
\end{frame}


\end{document}
